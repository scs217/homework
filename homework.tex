% !Rnw weave = knitr
% !TeX program = XeLaTeX

\documentclass[10pt]{article}\usepackage{graphicx, color}
%% maxwidth is the original width if it is less than linewidth
%% otherwise use linewidth (to make sure the graphics do not exceed the margin)
\makeatletter
\def\maxwidth{ %
  \ifdim\Gin@nat@width>\linewidth
    \linewidth
  \else
    \Gin@nat@width
  \fi
}
\makeatother

\definecolor{fgcolor}{rgb}{0.2, 0.2, 0.2}
\newcommand{\hlnumber}[1]{\textcolor[rgb]{0,0,0}{#1}}%
\newcommand{\hlfunctioncall}[1]{\textcolor[rgb]{0.501960784313725,0,0.329411764705882}{\textbf{#1}}}%
\newcommand{\hlstring}[1]{\textcolor[rgb]{0.6,0.6,1}{#1}}%
\newcommand{\hlkeyword}[1]{\textcolor[rgb]{0,0,0}{\textbf{#1}}}%
\newcommand{\hlargument}[1]{\textcolor[rgb]{0.690196078431373,0.250980392156863,0.0196078431372549}{#1}}%
\newcommand{\hlcomment}[1]{\textcolor[rgb]{0.180392156862745,0.6,0.341176470588235}{#1}}%
\newcommand{\hlroxygencomment}[1]{\textcolor[rgb]{0.43921568627451,0.47843137254902,0.701960784313725}{#1}}%
\newcommand{\hlformalargs}[1]{\textcolor[rgb]{0.690196078431373,0.250980392156863,0.0196078431372549}{#1}}%
\newcommand{\hleqformalargs}[1]{\textcolor[rgb]{0.690196078431373,0.250980392156863,0.0196078431372549}{#1}}%
\newcommand{\hlassignement}[1]{\textcolor[rgb]{0,0,0}{\textbf{#1}}}%
\newcommand{\hlpackage}[1]{\textcolor[rgb]{0.588235294117647,0.709803921568627,0.145098039215686}{#1}}%
\newcommand{\hlslot}[1]{\textit{#1}}%
\newcommand{\hlsymbol}[1]{\textcolor[rgb]{0,0,0}{#1}}%
\newcommand{\hlprompt}[1]{\textcolor[rgb]{0.2,0.2,0.2}{#1}}%

\usepackage{framed}
\makeatletter
\newenvironment{kframe}{%
 \def\at@end@of@kframe{}%
 \ifinner\ifhmode%
  \def\at@end@of@kframe{\end{minipage}}%
  \begin{minipage}{\columnwidth}%
 \fi\fi%
 \def\FrameCommand##1{\hskip\@totalleftmargin \hskip-\fboxsep
 \colorbox{shadecolor}{##1}\hskip-\fboxsep
     % There is no \\@totalrightmargin, so:
     \hskip-\linewidth \hskip-\@totalleftmargin \hskip\columnwidth}%
 \MakeFramed {\advance\hsize-\width
   \@totalleftmargin\z@ \linewidth\hsize
   \@setminipage}}%
 {\par\unskip\endMakeFramed%
 \at@end@of@kframe}
\makeatother

\definecolor{shadecolor}{rgb}{.97, .97, .97}
\definecolor{messagecolor}{rgb}{0, 0, 0}
\definecolor{warningcolor}{rgb}{1, 0, 1}
\definecolor{errorcolor}{rgb}{1, 0, 0}
\newenvironment{knitrout}{}{} % an empty environment to be redefined in TeX

\usepackage{alltt}

\usepackage{amsmath}
\usepackage[tmargin=1in,bmargin=1in,lmargin=1.25in,rmargin=1.25in]{geometry}
\usepackage{fontspec}
\usepackage{xcolor}
\usepackage{titlesec}
\defaultfontfeatures{Ligatures=TeX}
\usepackage{framed}

\setsansfont{Myriad Pro}
\setmainfont{Minion Pro}


\titleformat*{\section}{\Large\bfseries\sffamily}
\titleformat*{\subsection}{\large\bfseries\sffamily}
\titleformat*{\subsubsection}{\itshape\subsubsectionfont}

\title{Fortran 95: Finite Difference Solutions to Differential Equations}
\author{Shane Stevens}
\date{January 24, 2011}
\IfFileExists{upquote.sty}{\usepackage{upquote}}{}

\begin{document}

\maketitle

\tableofcontents

\listoffigures




\section{Chemical Kinetic Model of Consecutive Reactions}

\begin{eqnarray}
A \xrightarrow{k_1} B \\    
B \xrightarrow{k_2} C
\end{eqnarray}

\subsection{Differential Equations for above:}


\begin{eqnarray}
\frac{d[A]}{dt} & = -k_1[A] \\
\frac{d[B]}{dt} & = k_1[A] -k_2[B] \\
\frac{d[C]}{dt} & = k_2B
\end{eqnarray}


\subsection{Finite Difference Equations for Differential Equations:}


\begin{eqnarray}
\lbrack A(i) \rbrack & = [A(i-1)] -k_1[A(i-1)]\Delta t \\
\lbrack B(i)] & = [B(i-1)]+(k_1[A(i-1)]-k_2[B(i-1)]) \Delta t \\
\lbrack C(i)] & = [C(i-1)] + k_2[B(i-1)] \Delta t
\end{eqnarray}


\section{Fortran Code}

\begin{kframe}
\begin{alltt}
PROGRAM finite_differences
IMPLICIT NONE

DOUBLE PRECISION, \hlfunctioncall{DIMENSION} (2000,8) :: data=0.0
DOUBLE PRECISION, :: k1=5.0, k2=20.0, dt=0.005
INTEGER :: i, j

DO i=2, 2000
  \hlfunctioncall{data}(i,1) = \hlfunctioncall{data}(i-1,1) + \hlfunctioncall{data}(i-1,1)*(-k1)*dt
  \hlfunctioncall{data}(i,2) = \hlfunctioncall{data}(i-1,2) + (\hlfunctioncall{data}(i-1,1)*k1*dt - \hlfunctioncall{data}(i-1,2)*k2*dt)
  \hlfunctioncall{data}(i,3) = \hlfunctioncall{data}(i-1,3) + \hlfunctioncall{data}(i-1,2)*k2*dt
  \hlfunctioncall{data}(i,4) = \hlfunctioncall{data}(i-1,1)*k1/k2
  \hlfunctioncall{data}(i,5) = 500.0*\hlfunctioncall{DEXP}(-k1*i*dt)
  \hlfunctioncall{data}(i,6) = k1/(k2-k1)*5000.0*( \hlfunctioncall{DEXP}(-k1*i*dt) - \hlfunctioncall{DEXP}(-k2*i*dt) )
  \hlfunctioncall{data}(i,7) = 500.0*( 1- \hlfunctioncall{DEXP}(-k1*i*dt) - k1/(k2-k1) * (\hlfunctioncall{DEXP}(-k1*i*dt)-\hlfunctioncall{DEXP}(-k2*i*dt) ) )
  \hlfunctioncall{data}(i,8) = i*dt
END DO

\hlfunctioncall{OPEN}(UNIT=25, FILE=\hlstring{'output.csv'}, ACTION=\hlstring{'WRITE'}, STATUS=\hlstring{'REPLACE'})
\hlfunctioncall{WRITE}(25,30) ( (\hlfunctioncall{data}(i,j), j=1, 8), i=1, 2000)
30 \hlfunctioncall{FORMAT} ( \hlfunctioncall{7}(ES20.12, \hlstring{','}), ES20.12)

END PROGRAM finite_differences
\end{alltt}
\end{kframe}


\section{R Version of Program}

\begin{knitrout}
\definecolor{shadecolor}{rgb}{0.969, 0.969, 0.969}\color{fgcolor}\begin{kframe}
\begin{alltt}
\hlcomment{# Set Variables:}
k1 = 5
k2 = 20
dt = 5e-04
data = \hlfunctioncall{rep}(0, 16000)
\hlfunctioncall{dim}(data) = \hlfunctioncall{c}(2000, 8)
data = \hlfunctioncall{data.frame}(data)
data[1, 1] = 500

\hlfunctioncall{for} (i in 2:2000) \{
    data[i, 1] = data[i - 1, 1] - data[i - 1, 1] * k1 * dt
    data[i, 2] = data[i - 1, 2] + (data[i - 1, 1] * k1 * dt - data[i - 1, 2] * 
        k2 * dt)
    data[i, 3] = data[i - 1, 3] + data[i - 1, 2] * k2 * dt
    data[i, 4] = data[i - 1, 1] * k1/k2
    data[i, 5] = 500 * \hlfunctioncall{exp}(-k1 * i * dt)
    data[i, 6] = k1/(k2 - k1) * 500 * (\hlfunctioncall{exp}(-k1 * i * dt) - \hlfunctioncall{exp}(-k2 * i * dt))
    data[i, 7] = 500 * (1 - \hlfunctioncall{exp}(-k1 * i * dt) - k1/(k2 - k1) * (\hlfunctioncall{exp}(-k1 * i * 
        dt) - \hlfunctioncall{exp}(-k2 * i * dt)))
    data[i, 8] = i * dt
\}
\end{alltt}
\end{kframe}
\end{knitrout}


\begin{knitrout}
\definecolor{shadecolor}{rgb}{0.969, 0.969, 0.969}\color{fgcolor}\begin{kframe}
\begin{alltt}
\hlfunctioncall{names}(data)=\hlfunctioncall{c}(\hlstring{'A'},\hlstring{'B'},\hlstring{'C'},\hlstring{'Steady State'},\hlstring{'A Approx'},\hlstring{'B Approx'},\hlstring{'C Approx'},\hlstring{'Time'})
data2=\hlfunctioncall{melt}(data,id.vars=\hlstring{'Time'},
           measure.vars=\hlfunctioncall{c}(\hlstring{'A'},\hlstring{'B'},\hlstring{'C'},\hlstring{'Steady State'},\hlstring{'A Approx'},\hlstring{'B Approx'},\hlstring{'C Approx'}))
\hlfunctioncall{names}(data2)=\hlfunctioncall{c}(\hlstring{'Time'},\hlstring{'Trace'},\hlstring{'Concentration'})
\end{alltt}
\end{kframe}
\end{knitrout}



\begin{knitrout}
\definecolor{shadecolor}{rgb}{0.969, 0.969, 0.969}\color{fgcolor}\begin{kframe}
\begin{alltt}
\hlfunctioncall{ggplot}(data2,\hlfunctioncall{aes}(Time,Concentration,color=Trace,alpha=Trace))+
  \hlfunctioncall{geom_line}(\hlfunctioncall{aes}(linetype=Trace))+
  \hlfunctioncall{scale_linetype_manual}(values=\hlfunctioncall{c}(\hlfunctioncall{rep}(\hlstring{'solid'},3),\hlstring{'dotdash'},\hlfunctioncall{rep}(\hlstring{'dashed'},3)))+
  \hlfunctioncall{scale_color_manual}(values=\hlfunctioncall{c}(\hlstring{'red'},\hlstring{'blue'},\hlstring{'green'},\hlstring{'blue2'},\hlstring{'red'},\hlstring{'blue'},\hlstring{'green'}))+
  \hlfunctioncall{scale_alpha_manual}(values=\hlfunctioncall{c}(\hlfunctioncall{rep}(1,3),\hlfunctioncall{rep}(.4,4)))+\hlfunctioncall{theme_igray}()
\end{alltt}
\end{kframe}\begin{figure}[]

\includegraphics[width=\maxwidth]{figure/meltplots} \caption[Plot of R code generated above]{Plot of R code generated above\label{fig:meltplots}}
\end{figure}


\end{knitrout}



\end{document}
